\documentclass[11pt]{article}

    \usepackage[breakable]{tcolorbox}
    \usepackage{parskip} % Stop auto-indenting (to mimic markdown behaviour)
    
    \usepackage{iftex}
    \ifPDFTeX
    	\usepackage[T1]{fontenc}
    	\usepackage{mathpazo}
    \else
    	\usepackage{fontspec}
    \fi

    % Basic figure setup, for now with no caption control since it's done
    % automatically by Pandoc (which extracts ![](path) syntax from Markdown).
    \usepackage{graphicx}
    % Maintain compatibility with old templates. Remove in nbconvert 6.0
    \let\Oldincludegraphics\includegraphics
    % Ensure that by default, figures have no caption (until we provide a
    % proper Figure object with a Caption API and a way to capture that
    % in the conversion process - todo).
    \usepackage{caption}
    \DeclareCaptionFormat{nocaption}{}
    \captionsetup{format=nocaption,aboveskip=0pt,belowskip=0pt}

    \usepackage[Export]{adjustbox} % Used to constrain images to a maximum size
    \adjustboxset{max size={0.9\linewidth}{0.9\paperheight}}
    \usepackage{float}
    \floatplacement{figure}{H} % forces figures to be placed at the correct location
    \usepackage{xcolor} % Allow colors to be defined
    \usepackage{enumerate} % Needed for markdown enumerations to work
    \usepackage{geometry} % Used to adjust the document margins
    \usepackage{amsmath} % Equations
    \usepackage{amssymb} % Equations
    \usepackage{textcomp} % defines textquotesingle
    % Hack from http://tex.stackexchange.com/a/47451/13684:
    \AtBeginDocument{%
        \def\PYZsq{\textquotesingle}% Upright quotes in Pygmentized code
    }
    \usepackage{upquote} % Upright quotes for verbatim code
    \usepackage{eurosym} % defines \euro
    \usepackage[mathletters]{ucs} % Extended unicode (utf-8) support
    \usepackage{fancyvrb} % verbatim replacement that allows latex
    \usepackage{grffile} % extends the file name processing of package graphics 
                         % to support a larger range
    \makeatletter % fix for grffile with XeLaTeX
    \def\Gread@@xetex#1{%
      \IfFileExists{"\Gin@base".bb}%
      {\Gread@eps{\Gin@base.bb}}%
      {\Gread@@xetex@aux#1}%
    }
    \makeatother

    % The hyperref package gives us a pdf with properly built
    % internal navigation ('pdf bookmarks' for the table of contents,
    % internal cross-reference links, web links for URLs, etc.)
    \usepackage{hyperref}
    % The default LaTeX title has an obnoxious amount of whitespace. By default,
    % titling removes some of it. It also provides customization options.
    \usepackage{titling}
    \usepackage{longtable} % longtable support required by pandoc >1.10
    \usepackage{booktabs}  % table support for pandoc > 1.12.2
    \usepackage[inline]{enumitem} % IRkernel/repr support (it uses the enumerate* environment)
    \usepackage[normalem]{ulem} % ulem is needed to support strikethroughs (\sout)
                                % normalem makes italics be italics, not underlines
    \usepackage{mathrsfs}
    

    
    % Colors for the hyperref package
    \definecolor{urlcolor}{rgb}{0,.145,.698}
    \definecolor{linkcolor}{rgb}{.71,0.21,0.01}
    \definecolor{citecolor}{rgb}{.12,.54,.11}

    % ANSI colors
    \definecolor{ansi-black}{HTML}{3E424D}
    \definecolor{ansi-black-intense}{HTML}{282C36}
    \definecolor{ansi-red}{HTML}{E75C58}
    \definecolor{ansi-red-intense}{HTML}{B22B31}
    \definecolor{ansi-green}{HTML}{00A250}
    \definecolor{ansi-green-intense}{HTML}{007427}
    \definecolor{ansi-yellow}{HTML}{DDB62B}
    \definecolor{ansi-yellow-intense}{HTML}{B27D12}
    \definecolor{ansi-blue}{HTML}{208FFB}
    \definecolor{ansi-blue-intense}{HTML}{0065CA}
    \definecolor{ansi-magenta}{HTML}{D160C4}
    \definecolor{ansi-magenta-intense}{HTML}{A03196}
    \definecolor{ansi-cyan}{HTML}{60C6C8}
    \definecolor{ansi-cyan-intense}{HTML}{258F8F}
    \definecolor{ansi-white}{HTML}{C5C1B4}
    \definecolor{ansi-white-intense}{HTML}{A1A6B2}
    \definecolor{ansi-default-inverse-fg}{HTML}{FFFFFF}
    \definecolor{ansi-default-inverse-bg}{HTML}{000000}

    % commands and environments needed by pandoc snippets
    % extracted from the output of `pandoc -s`
    \providecommand{\tightlist}{%
      \setlength{\itemsep}{0pt}\setlength{\parskip}{0pt}}
    \DefineVerbatimEnvironment{Highlighting}{Verbatim}{commandchars=\\\{\}}
    % Add ',fontsize=\small' for more characters per line
    \newenvironment{Shaded}{}{}
    \newcommand{\KeywordTok}[1]{\textcolor[rgb]{0.00,0.44,0.13}{\textbf{{#1}}}}
    \newcommand{\DataTypeTok}[1]{\textcolor[rgb]{0.56,0.13,0.00}{{#1}}}
    \newcommand{\DecValTok}[1]{\textcolor[rgb]{0.25,0.63,0.44}{{#1}}}
    \newcommand{\BaseNTok}[1]{\textcolor[rgb]{0.25,0.63,0.44}{{#1}}}
    \newcommand{\FloatTok}[1]{\textcolor[rgb]{0.25,0.63,0.44}{{#1}}}
    \newcommand{\CharTok}[1]{\textcolor[rgb]{0.25,0.44,0.63}{{#1}}}
    \newcommand{\StringTok}[1]{\textcolor[rgb]{0.25,0.44,0.63}{{#1}}}
    \newcommand{\CommentTok}[1]{\textcolor[rgb]{0.38,0.63,0.69}{\textit{{#1}}}}
    \newcommand{\OtherTok}[1]{\textcolor[rgb]{0.00,0.44,0.13}{{#1}}}
    \newcommand{\AlertTok}[1]{\textcolor[rgb]{1.00,0.00,0.00}{\textbf{{#1}}}}
    \newcommand{\FunctionTok}[1]{\textcolor[rgb]{0.02,0.16,0.49}{{#1}}}
    \newcommand{\RegionMarkerTok}[1]{{#1}}
    \newcommand{\ErrorTok}[1]{\textcolor[rgb]{1.00,0.00,0.00}{\textbf{{#1}}}}
    \newcommand{\NormalTok}[1]{{#1}}
    
    % Additional commands for more recent versions of Pandoc
    \newcommand{\ConstantTok}[1]{\textcolor[rgb]{0.53,0.00,0.00}{{#1}}}
    \newcommand{\SpecialCharTok}[1]{\textcolor[rgb]{0.25,0.44,0.63}{{#1}}}
    \newcommand{\VerbatimStringTok}[1]{\textcolor[rgb]{0.25,0.44,0.63}{{#1}}}
    \newcommand{\SpecialStringTok}[1]{\textcolor[rgb]{0.73,0.40,0.53}{{#1}}}
    \newcommand{\ImportTok}[1]{{#1}}
    \newcommand{\DocumentationTok}[1]{\textcolor[rgb]{0.73,0.13,0.13}{\textit{{#1}}}}
    \newcommand{\AnnotationTok}[1]{\textcolor[rgb]{0.38,0.63,0.69}{\textbf{\textit{{#1}}}}}
    \newcommand{\CommentVarTok}[1]{\textcolor[rgb]{0.38,0.63,0.69}{\textbf{\textit{{#1}}}}}
    \newcommand{\VariableTok}[1]{\textcolor[rgb]{0.10,0.09,0.49}{{#1}}}
    \newcommand{\ControlFlowTok}[1]{\textcolor[rgb]{0.00,0.44,0.13}{\textbf{{#1}}}}
    \newcommand{\OperatorTok}[1]{\textcolor[rgb]{0.40,0.40,0.40}{{#1}}}
    \newcommand{\BuiltInTok}[1]{{#1}}
    \newcommand{\ExtensionTok}[1]{{#1}}
    \newcommand{\PreprocessorTok}[1]{\textcolor[rgb]{0.74,0.48,0.00}{{#1}}}
    \newcommand{\AttributeTok}[1]{\textcolor[rgb]{0.49,0.56,0.16}{{#1}}}
    \newcommand{\InformationTok}[1]{\textcolor[rgb]{0.38,0.63,0.69}{\textbf{\textit{{#1}}}}}
    \newcommand{\WarningTok}[1]{\textcolor[rgb]{0.38,0.63,0.69}{\textbf{\textit{{#1}}}}}
    
    
    % Define a nice break command that doesn't care if a line doesn't already
    % exist.
    \def\br{\hspace*{\fill} \\* }
    % Math Jax compatibility definitions
    \def\gt{>}
    \def\lt{<}
    \let\Oldtex\TeX
    \let\Oldlatex\LaTeX
    \renewcommand{\TeX}{\textrm{\Oldtex}}
    \renewcommand{\LaTeX}{\textrm{\Oldlatex}}
    % Document parameters
    % Document title
    \title{updated\_exact\_optsol\_intLP\_Copy}
    
    
    
    
    
% Pygments definitions
\makeatletter
\def\PY@reset{\let\PY@it=\relax \let\PY@bf=\relax%
    \let\PY@ul=\relax \let\PY@tc=\relax%
    \let\PY@bc=\relax \let\PY@ff=\relax}
\def\PY@tok#1{\csname PY@tok@#1\endcsname}
\def\PY@toks#1+{\ifx\relax#1\empty\else%
    \PY@tok{#1}\expandafter\PY@toks\fi}
\def\PY@do#1{\PY@bc{\PY@tc{\PY@ul{%
    \PY@it{\PY@bf{\PY@ff{#1}}}}}}}
\def\PY#1#2{\PY@reset\PY@toks#1+\relax+\PY@do{#2}}

\expandafter\def\csname PY@tok@w\endcsname{\def\PY@tc##1{\textcolor[rgb]{0.73,0.73,0.73}{##1}}}
\expandafter\def\csname PY@tok@c\endcsname{\let\PY@it=\textit\def\PY@tc##1{\textcolor[rgb]{0.25,0.50,0.50}{##1}}}
\expandafter\def\csname PY@tok@cp\endcsname{\def\PY@tc##1{\textcolor[rgb]{0.74,0.48,0.00}{##1}}}
\expandafter\def\csname PY@tok@k\endcsname{\let\PY@bf=\textbf\def\PY@tc##1{\textcolor[rgb]{0.00,0.50,0.00}{##1}}}
\expandafter\def\csname PY@tok@kp\endcsname{\def\PY@tc##1{\textcolor[rgb]{0.00,0.50,0.00}{##1}}}
\expandafter\def\csname PY@tok@kt\endcsname{\def\PY@tc##1{\textcolor[rgb]{0.69,0.00,0.25}{##1}}}
\expandafter\def\csname PY@tok@o\endcsname{\def\PY@tc##1{\textcolor[rgb]{0.40,0.40,0.40}{##1}}}
\expandafter\def\csname PY@tok@ow\endcsname{\let\PY@bf=\textbf\def\PY@tc##1{\textcolor[rgb]{0.67,0.13,1.00}{##1}}}
\expandafter\def\csname PY@tok@nb\endcsname{\def\PY@tc##1{\textcolor[rgb]{0.00,0.50,0.00}{##1}}}
\expandafter\def\csname PY@tok@nf\endcsname{\def\PY@tc##1{\textcolor[rgb]{0.00,0.00,1.00}{##1}}}
\expandafter\def\csname PY@tok@nc\endcsname{\let\PY@bf=\textbf\def\PY@tc##1{\textcolor[rgb]{0.00,0.00,1.00}{##1}}}
\expandafter\def\csname PY@tok@nn\endcsname{\let\PY@bf=\textbf\def\PY@tc##1{\textcolor[rgb]{0.00,0.00,1.00}{##1}}}
\expandafter\def\csname PY@tok@ne\endcsname{\let\PY@bf=\textbf\def\PY@tc##1{\textcolor[rgb]{0.82,0.25,0.23}{##1}}}
\expandafter\def\csname PY@tok@nv\endcsname{\def\PY@tc##1{\textcolor[rgb]{0.10,0.09,0.49}{##1}}}
\expandafter\def\csname PY@tok@no\endcsname{\def\PY@tc##1{\textcolor[rgb]{0.53,0.00,0.00}{##1}}}
\expandafter\def\csname PY@tok@nl\endcsname{\def\PY@tc##1{\textcolor[rgb]{0.63,0.63,0.00}{##1}}}
\expandafter\def\csname PY@tok@ni\endcsname{\let\PY@bf=\textbf\def\PY@tc##1{\textcolor[rgb]{0.60,0.60,0.60}{##1}}}
\expandafter\def\csname PY@tok@na\endcsname{\def\PY@tc##1{\textcolor[rgb]{0.49,0.56,0.16}{##1}}}
\expandafter\def\csname PY@tok@nt\endcsname{\let\PY@bf=\textbf\def\PY@tc##1{\textcolor[rgb]{0.00,0.50,0.00}{##1}}}
\expandafter\def\csname PY@tok@nd\endcsname{\def\PY@tc##1{\textcolor[rgb]{0.67,0.13,1.00}{##1}}}
\expandafter\def\csname PY@tok@s\endcsname{\def\PY@tc##1{\textcolor[rgb]{0.73,0.13,0.13}{##1}}}
\expandafter\def\csname PY@tok@sd\endcsname{\let\PY@it=\textit\def\PY@tc##1{\textcolor[rgb]{0.73,0.13,0.13}{##1}}}
\expandafter\def\csname PY@tok@si\endcsname{\let\PY@bf=\textbf\def\PY@tc##1{\textcolor[rgb]{0.73,0.40,0.53}{##1}}}
\expandafter\def\csname PY@tok@se\endcsname{\let\PY@bf=\textbf\def\PY@tc##1{\textcolor[rgb]{0.73,0.40,0.13}{##1}}}
\expandafter\def\csname PY@tok@sr\endcsname{\def\PY@tc##1{\textcolor[rgb]{0.73,0.40,0.53}{##1}}}
\expandafter\def\csname PY@tok@ss\endcsname{\def\PY@tc##1{\textcolor[rgb]{0.10,0.09,0.49}{##1}}}
\expandafter\def\csname PY@tok@sx\endcsname{\def\PY@tc##1{\textcolor[rgb]{0.00,0.50,0.00}{##1}}}
\expandafter\def\csname PY@tok@m\endcsname{\def\PY@tc##1{\textcolor[rgb]{0.40,0.40,0.40}{##1}}}
\expandafter\def\csname PY@tok@gh\endcsname{\let\PY@bf=\textbf\def\PY@tc##1{\textcolor[rgb]{0.00,0.00,0.50}{##1}}}
\expandafter\def\csname PY@tok@gu\endcsname{\let\PY@bf=\textbf\def\PY@tc##1{\textcolor[rgb]{0.50,0.00,0.50}{##1}}}
\expandafter\def\csname PY@tok@gd\endcsname{\def\PY@tc##1{\textcolor[rgb]{0.63,0.00,0.00}{##1}}}
\expandafter\def\csname PY@tok@gi\endcsname{\def\PY@tc##1{\textcolor[rgb]{0.00,0.63,0.00}{##1}}}
\expandafter\def\csname PY@tok@gr\endcsname{\def\PY@tc##1{\textcolor[rgb]{1.00,0.00,0.00}{##1}}}
\expandafter\def\csname PY@tok@ge\endcsname{\let\PY@it=\textit}
\expandafter\def\csname PY@tok@gs\endcsname{\let\PY@bf=\textbf}
\expandafter\def\csname PY@tok@gp\endcsname{\let\PY@bf=\textbf\def\PY@tc##1{\textcolor[rgb]{0.00,0.00,0.50}{##1}}}
\expandafter\def\csname PY@tok@go\endcsname{\def\PY@tc##1{\textcolor[rgb]{0.53,0.53,0.53}{##1}}}
\expandafter\def\csname PY@tok@gt\endcsname{\def\PY@tc##1{\textcolor[rgb]{0.00,0.27,0.87}{##1}}}
\expandafter\def\csname PY@tok@err\endcsname{\def\PY@bc##1{\setlength{\fboxsep}{0pt}\fcolorbox[rgb]{1.00,0.00,0.00}{1,1,1}{\strut ##1}}}
\expandafter\def\csname PY@tok@kc\endcsname{\let\PY@bf=\textbf\def\PY@tc##1{\textcolor[rgb]{0.00,0.50,0.00}{##1}}}
\expandafter\def\csname PY@tok@kd\endcsname{\let\PY@bf=\textbf\def\PY@tc##1{\textcolor[rgb]{0.00,0.50,0.00}{##1}}}
\expandafter\def\csname PY@tok@kn\endcsname{\let\PY@bf=\textbf\def\PY@tc##1{\textcolor[rgb]{0.00,0.50,0.00}{##1}}}
\expandafter\def\csname PY@tok@kr\endcsname{\let\PY@bf=\textbf\def\PY@tc##1{\textcolor[rgb]{0.00,0.50,0.00}{##1}}}
\expandafter\def\csname PY@tok@bp\endcsname{\def\PY@tc##1{\textcolor[rgb]{0.00,0.50,0.00}{##1}}}
\expandafter\def\csname PY@tok@fm\endcsname{\def\PY@tc##1{\textcolor[rgb]{0.00,0.00,1.00}{##1}}}
\expandafter\def\csname PY@tok@vc\endcsname{\def\PY@tc##1{\textcolor[rgb]{0.10,0.09,0.49}{##1}}}
\expandafter\def\csname PY@tok@vg\endcsname{\def\PY@tc##1{\textcolor[rgb]{0.10,0.09,0.49}{##1}}}
\expandafter\def\csname PY@tok@vi\endcsname{\def\PY@tc##1{\textcolor[rgb]{0.10,0.09,0.49}{##1}}}
\expandafter\def\csname PY@tok@vm\endcsname{\def\PY@tc##1{\textcolor[rgb]{0.10,0.09,0.49}{##1}}}
\expandafter\def\csname PY@tok@sa\endcsname{\def\PY@tc##1{\textcolor[rgb]{0.73,0.13,0.13}{##1}}}
\expandafter\def\csname PY@tok@sb\endcsname{\def\PY@tc##1{\textcolor[rgb]{0.73,0.13,0.13}{##1}}}
\expandafter\def\csname PY@tok@sc\endcsname{\def\PY@tc##1{\textcolor[rgb]{0.73,0.13,0.13}{##1}}}
\expandafter\def\csname PY@tok@dl\endcsname{\def\PY@tc##1{\textcolor[rgb]{0.73,0.13,0.13}{##1}}}
\expandafter\def\csname PY@tok@s2\endcsname{\def\PY@tc##1{\textcolor[rgb]{0.73,0.13,0.13}{##1}}}
\expandafter\def\csname PY@tok@sh\endcsname{\def\PY@tc##1{\textcolor[rgb]{0.73,0.13,0.13}{##1}}}
\expandafter\def\csname PY@tok@s1\endcsname{\def\PY@tc##1{\textcolor[rgb]{0.73,0.13,0.13}{##1}}}
\expandafter\def\csname PY@tok@mb\endcsname{\def\PY@tc##1{\textcolor[rgb]{0.40,0.40,0.40}{##1}}}
\expandafter\def\csname PY@tok@mf\endcsname{\def\PY@tc##1{\textcolor[rgb]{0.40,0.40,0.40}{##1}}}
\expandafter\def\csname PY@tok@mh\endcsname{\def\PY@tc##1{\textcolor[rgb]{0.40,0.40,0.40}{##1}}}
\expandafter\def\csname PY@tok@mi\endcsname{\def\PY@tc##1{\textcolor[rgb]{0.40,0.40,0.40}{##1}}}
\expandafter\def\csname PY@tok@il\endcsname{\def\PY@tc##1{\textcolor[rgb]{0.40,0.40,0.40}{##1}}}
\expandafter\def\csname PY@tok@mo\endcsname{\def\PY@tc##1{\textcolor[rgb]{0.40,0.40,0.40}{##1}}}
\expandafter\def\csname PY@tok@ch\endcsname{\let\PY@it=\textit\def\PY@tc##1{\textcolor[rgb]{0.25,0.50,0.50}{##1}}}
\expandafter\def\csname PY@tok@cm\endcsname{\let\PY@it=\textit\def\PY@tc##1{\textcolor[rgb]{0.25,0.50,0.50}{##1}}}
\expandafter\def\csname PY@tok@cpf\endcsname{\let\PY@it=\textit\def\PY@tc##1{\textcolor[rgb]{0.25,0.50,0.50}{##1}}}
\expandafter\def\csname PY@tok@c1\endcsname{\let\PY@it=\textit\def\PY@tc##1{\textcolor[rgb]{0.25,0.50,0.50}{##1}}}
\expandafter\def\csname PY@tok@cs\endcsname{\let\PY@it=\textit\def\PY@tc##1{\textcolor[rgb]{0.25,0.50,0.50}{##1}}}

\def\PYZbs{\char`\\}
\def\PYZus{\char`\_}
\def\PYZob{\char`\{}
\def\PYZcb{\char`\}}
\def\PYZca{\char`\^}
\def\PYZam{\char`\&}
\def\PYZlt{\char`\<}
\def\PYZgt{\char`\>}
\def\PYZsh{\char`\#}
\def\PYZpc{\char`\%}
\def\PYZdl{\char`\$}
\def\PYZhy{\char`\-}
\def\PYZsq{\char`\'}
\def\PYZdq{\char`\"}
\def\PYZti{\char`\~}
% for compatibility with earlier versions
\def\PYZat{@}
\def\PYZlb{[}
\def\PYZrb{]}
\makeatother


    % For linebreaks inside Verbatim environment from package fancyvrb. 
    \makeatletter
        \newbox\Wrappedcontinuationbox 
        \newbox\Wrappedvisiblespacebox 
        \newcommand*\Wrappedvisiblespace {\textcolor{red}{\textvisiblespace}} 
        \newcommand*\Wrappedcontinuationsymbol {\textcolor{red}{\llap{\tiny$\m@th\hookrightarrow$}}} 
        \newcommand*\Wrappedcontinuationindent {3ex } 
        \newcommand*\Wrappedafterbreak {\kern\Wrappedcontinuationindent\copy\Wrappedcontinuationbox} 
        % Take advantage of the already applied Pygments mark-up to insert 
        % potential linebreaks for TeX processing. 
        %        {, <, #, %, $, ' and ": go to next line. 
        %        _, }, ^, &, >, - and ~: stay at end of broken line. 
        % Use of \textquotesingle for straight quote. 
        \newcommand*\Wrappedbreaksatspecials {% 
            \def\PYGZus{\discretionary{\char`\_}{\Wrappedafterbreak}{\char`\_}}% 
            \def\PYGZob{\discretionary{}{\Wrappedafterbreak\char`\{}{\char`\{}}% 
            \def\PYGZcb{\discretionary{\char`\}}{\Wrappedafterbreak}{\char`\}}}% 
            \def\PYGZca{\discretionary{\char`\^}{\Wrappedafterbreak}{\char`\^}}% 
            \def\PYGZam{\discretionary{\char`\&}{\Wrappedafterbreak}{\char`\&}}% 
            \def\PYGZlt{\discretionary{}{\Wrappedafterbreak\char`\<}{\char`\<}}% 
            \def\PYGZgt{\discretionary{\char`\>}{\Wrappedafterbreak}{\char`\>}}% 
            \def\PYGZsh{\discretionary{}{\Wrappedafterbreak\char`\#}{\char`\#}}% 
            \def\PYGZpc{\discretionary{}{\Wrappedafterbreak\char`\%}{\char`\%}}% 
            \def\PYGZdl{\discretionary{}{\Wrappedafterbreak\char`\$}{\char`\$}}% 
            \def\PYGZhy{\discretionary{\char`\-}{\Wrappedafterbreak}{\char`\-}}% 
            \def\PYGZsq{\discretionary{}{\Wrappedafterbreak\textquotesingle}{\textquotesingle}}% 
            \def\PYGZdq{\discretionary{}{\Wrappedafterbreak\char`\"}{\char`\"}}% 
            \def\PYGZti{\discretionary{\char`\~}{\Wrappedafterbreak}{\char`\~}}% 
        } 
        % Some characters . , ; ? ! / are not pygmentized. 
        % This macro makes them "active" and they will insert potential linebreaks 
        \newcommand*\Wrappedbreaksatpunct {% 
            \lccode`\~`\.\lowercase{\def~}{\discretionary{\hbox{\char`\.}}{\Wrappedafterbreak}{\hbox{\char`\.}}}% 
            \lccode`\~`\,\lowercase{\def~}{\discretionary{\hbox{\char`\,}}{\Wrappedafterbreak}{\hbox{\char`\,}}}% 
            \lccode`\~`\;\lowercase{\def~}{\discretionary{\hbox{\char`\;}}{\Wrappedafterbreak}{\hbox{\char`\;}}}% 
            \lccode`\~`\:\lowercase{\def~}{\discretionary{\hbox{\char`\:}}{\Wrappedafterbreak}{\hbox{\char`\:}}}% 
            \lccode`\~`\?\lowercase{\def~}{\discretionary{\hbox{\char`\?}}{\Wrappedafterbreak}{\hbox{\char`\?}}}% 
            \lccode`\~`\!\lowercase{\def~}{\discretionary{\hbox{\char`\!}}{\Wrappedafterbreak}{\hbox{\char`\!}}}% 
            \lccode`\~`\/\lowercase{\def~}{\discretionary{\hbox{\char`\/}}{\Wrappedafterbreak}{\hbox{\char`\/}}}% 
            \catcode`\.\active
            \catcode`\,\active 
            \catcode`\;\active
            \catcode`\:\active
            \catcode`\?\active
            \catcode`\!\active
            \catcode`\/\active 
            \lccode`\~`\~ 	
        }
    \makeatother

    \let\OriginalVerbatim=\Verbatim
    \makeatletter
    \renewcommand{\Verbatim}[1][1]{%
        %\parskip\z@skip
        \sbox\Wrappedcontinuationbox {\Wrappedcontinuationsymbol}%
        \sbox\Wrappedvisiblespacebox {\FV@SetupFont\Wrappedvisiblespace}%
        \def\FancyVerbFormatLine ##1{\hsize\linewidth
            \vtop{\raggedright\hyphenpenalty\z@\exhyphenpenalty\z@
                \doublehyphendemerits\z@\finalhyphendemerits\z@
                \strut ##1\strut}%
        }%
        % If the linebreak is at a space, the latter will be displayed as visible
        % space at end of first line, and a continuation symbol starts next line.
        % Stretch/shrink are however usually zero for typewriter font.
        \def\FV@Space {%
            \nobreak\hskip\z@ plus\fontdimen3\font minus\fontdimen4\font
            \discretionary{\copy\Wrappedvisiblespacebox}{\Wrappedafterbreak}
            {\kern\fontdimen2\font}%
        }%
        
        % Allow breaks at special characters using \PYG... macros.
        \Wrappedbreaksatspecials
        % Breaks at punctuation characters . , ; ? ! and / need catcode=\active 	
        \OriginalVerbatim[#1,codes*=\Wrappedbreaksatpunct]%
    }
    \makeatother

    % Exact colors from NB
    \definecolor{incolor}{HTML}{303F9F}
    \definecolor{outcolor}{HTML}{D84315}
    \definecolor{cellborder}{HTML}{CFCFCF}
    \definecolor{cellbackground}{HTML}{F7F7F7}
    
    % prompt
    \makeatletter
    \newcommand{\boxspacing}{\kern\kvtcb@left@rule\kern\kvtcb@boxsep}
    \makeatother
    \newcommand{\prompt}[4]{
        \ttfamily\llap{{\color{#2}[#3]:\hspace{3pt}#4}}\vspace{-\baselineskip}
    }
    

    
    % Prevent overflowing lines due to hard-to-break entities
    \sloppy 
    % Setup hyperref package
    \hypersetup{
      breaklinks=true,  % so long urls are correctly broken across lines
      colorlinks=true,
      urlcolor=urlcolor,
      linkcolor=linkcolor,
      citecolor=citecolor,
      }
    % Slightly bigger margins than the latex defaults
    
    \geometry{verbose,tmargin=1in,bmargin=1in,lmargin=1in,rmargin=1in}
    
    

\begin{document}
    
    \maketitle
    
    

    
    \begin{tcolorbox}[breakable, size=fbox, boxrule=1pt, pad at break*=1mm,colback=cellbackground, colframe=cellborder]
\prompt{In}{incolor}{1}{\boxspacing}
\begin{Verbatim}[commandchars=\\\{\}]
\PY{c+c1}{\PYZsh{} input matrix A should be the same as program\PYZhy{}reconstructed matrix LP\PYZus{}A}
\PY{c+c1}{\PYZsh{} here, since A has at least one \PYZdq{}0\PYZdq{} entry, LP.constraints() does not display this}
\PY{c+c1}{\PYZsh{} and so LP\PYZus{}A does not record the \PYZdq{}0\PYZdq{} entry, resulting in LP\PYZus{}A being non\PYZhy{}square}

\PY{k+kn}{from} \PY{n+nn}{sage}\PY{n+nn}{.}\PY{n+nn}{numerical}\PY{n+nn}{.}\PY{n+nn}{interactive\PYZus{}simplex\PYZus{}method} \PY{k}{import} \PY{o}{*}
\PY{k+kn}{from} \PY{n+nn}{sage}\PY{n+nn}{.}\PY{n+nn}{numerical}\PY{n+nn}{.}\PY{n+nn}{backends}\PY{n+nn}{.}\PY{n+nn}{generic\PYZus{}backend} \PY{k}{import} \PY{n}{get\PYZus{}solver}

\PY{n}{c} \PY{o}{=} \PY{n}{vector}\PY{p}{(}\PY{p}{(}\PY{l+m+mi}{91}\PY{o}{/}\PY{l+m+mi}{53}\PY{p}{,} \PY{l+m+mi}{1}\PY{o}{/}\PY{l+m+mi}{100}\PY{p}{,} \PY{l+m+mi}{1}\PY{o}{/}\PY{l+m+mi}{100}\PY{p}{,} \PY{l+m+mi}{1}\PY{o}{/}\PY{l+m+mi}{100}\PY{p}{,} \PY{l+m+mi}{1}\PY{o}{/}\PY{l+m+mi}{100}\PY{p}{,} \PY{l+m+mi}{1}\PY{o}{/}\PY{l+m+mi}{100}\PY{p}{,} \PY{l+m+mi}{1}\PY{o}{/}\PY{l+m+mi}{9}\PY{p}{,} \PY{l+m+mi}{3}\PY{o}{/}\PY{l+m+mi}{44}\PY{p}{,} \PY{l+m+mi}{17}\PY{o}{/}\PY{l+m+mi}{75}\PY{p}{,} \PY{l+m+mi}{1}\PY{o}{/}\PY{l+m+mi}{100}\PY{p}{)}\PY{p}{)}
\PY{n}{Y} \PY{o}{=} \PY{n}{vector}\PY{p}{(}\PY{p}{(}\PY{l+m+mi}{5}\PY{o}{/}\PY{l+m+mi}{68}\PY{p}{,} \PY{l+m+mi}{1}\PY{o}{/}\PY{l+m+mi}{100}\PY{p}{,} \PY{l+m+mi}{29}\PY{o}{/}\PY{l+m+mi}{36}\PY{p}{,} \PY{l+m+mi}{37}\PY{o}{/}\PY{l+m+mi}{14}\PY{p}{,} \PY{l+m+mi}{67}\PY{o}{/}\PY{l+m+mi}{18}\PY{p}{,} \PY{l+m+mi}{1}\PY{o}{/}\PY{l+m+mi}{100}\PY{p}{,} \PY{l+m+mi}{11}\PY{o}{/}\PY{l+m+mi}{21}\PY{p}{,} \PY{l+m+mi}{1}\PY{o}{/}\PY{l+m+mi}{100}\PY{p}{,} \PY{l+m+mi}{44}\PY{o}{/}\PY{l+m+mi}{83}\PY{p}{,} \PY{l+m+mi}{5}\PY{o}{/}\PY{l+m+mi}{38}\PY{p}{)}\PY{p}{)}
\PY{n}{A} \PY{o}{=} \PY{n}{Matrix}\PY{p}{(}\PY{p}{[}\PY{p}{[} \PY{l+m+mi}{13}\PY{o}{/}\PY{l+m+mi}{17}\PY{p}{,}  \PY{l+m+mi}{84}\PY{o}{/}\PY{l+m+mi}{95}\PY{p}{,} \PY{o}{\PYZhy{}}\PY{l+m+mi}{39}\PY{o}{/}\PY{l+m+mi}{46}\PY{p}{,}  \PY{l+m+mi}{72}\PY{o}{/}\PY{l+m+mi}{65}\PY{p}{,} \PY{o}{\PYZhy{}}\PY{l+m+mi}{45}\PY{o}{/}\PY{l+m+mi}{16}\PY{p}{,}  \PY{o}{\PYZhy{}}\PY{l+m+mi}{43}\PY{o}{/}\PY{l+m+mi}{6}\PY{p}{,}  \PY{l+m+mi}{23}\PY{o}{/}\PY{l+m+mi}{57}\PY{p}{,}  \PY{l+m+mi}{11}\PY{o}{/}\PY{l+m+mi}{15}\PY{p}{,}  \PY{l+m+mi}{64}\PY{o}{/}\PY{l+m+mi}{97}\PY{p}{,} \PY{o}{\PYZhy{}}\PY{l+m+mi}{49}\PY{o}{/}\PY{l+m+mi}{26}\PY{p}{]}\PY{p}{,}
\PY{p}{[}  \PY{l+m+mi}{82}\PY{o}{/}\PY{l+m+mi}{9}\PY{p}{,} \PY{o}{\PYZhy{}}\PY{l+m+mi}{73}\PY{o}{/}\PY{l+m+mi}{42}\PY{p}{,} \PY{o}{\PYZhy{}}\PY{l+m+mi}{55}\PY{o}{/}\PY{l+m+mi}{58}\PY{p}{,}   \PY{o}{\PYZhy{}}\PY{l+m+mi}{8}\PY{o}{/}\PY{l+m+mi}{5}\PY{p}{,}  \PY{l+m+mi}{57}\PY{o}{/}\PY{l+m+mi}{23}\PY{p}{,}  \PY{l+m+mi}{53}\PY{o}{/}\PY{l+m+mi}{51}\PY{p}{,}  \PY{l+m+mi}{77}\PY{o}{/}\PY{l+m+mi}{19}\PY{p}{,}  \PY{l+m+mi}{65}\PY{o}{/}\PY{l+m+mi}{28}\PY{p}{,}   \PY{l+m+mi}{32}\PY{o}{/}\PY{l+m+mi}{3}\PY{p}{,} \PY{o}{\PYZhy{}}\PY{l+m+mi}{11}\PY{o}{/}\PY{l+m+mi}{21}\PY{p}{]}\PY{p}{,}
\PY{p}{[} \PY{l+m+mi}{41}\PY{o}{/}\PY{l+m+mi}{14}\PY{p}{,}  \PY{l+m+mi}{100}\PY{o}{/}\PY{l+m+mi}{3}\PY{p}{,} \PY{o}{\PYZhy{}}\PY{l+m+mi}{86}\PY{o}{/}\PY{l+m+mi}{75}\PY{p}{,}   \PY{o}{\PYZhy{}}\PY{l+m+mi}{4}\PY{o}{/}\PY{l+m+mi}{3}\PY{p}{,}  \PY{l+m+mi}{87}\PY{o}{/}\PY{l+m+mi}{65}\PY{p}{,} \PY{o}{\PYZhy{}}\PY{l+m+mi}{97}\PY{o}{/}\PY{l+m+mi}{82}\PY{p}{,}  \PY{l+m+mi}{34}\PY{o}{/}\PY{l+m+mi}{81}\PY{p}{,}      \PY{l+m+mi}{6}\PY{p}{,}      \PY{l+m+mi}{1}\PY{p}{,} \PY{o}{\PYZhy{}}\PY{l+m+mi}{96}\PY{o}{/}\PY{l+m+mi}{17}\PY{p}{]}\PY{p}{,}
\PY{p}{[}  \PY{l+m+mi}{97}\PY{o}{/}\PY{l+m+mi}{2}\PY{p}{,}  \PY{o}{\PYZhy{}}\PY{l+m+mi}{23}\PY{o}{/}\PY{l+m+mi}{5}\PY{p}{,} \PY{o}{\PYZhy{}}\PY{l+m+mi}{26}\PY{o}{/}\PY{l+m+mi}{45}\PY{p}{,}  \PY{l+m+mi}{13}\PY{o}{/}\PY{l+m+mi}{28}\PY{p}{,} \PY{o}{\PYZhy{}}\PY{l+m+mi}{21}\PY{o}{/}\PY{l+m+mi}{19}\PY{p}{,} \PY{o}{\PYZhy{}}\PY{l+m+mi}{32}\PY{o}{/}\PY{l+m+mi}{13}\PY{p}{,} \PY{o}{\PYZhy{}}\PY{l+m+mi}{47}\PY{o}{/}\PY{l+m+mi}{43}\PY{p}{,} \PY{o}{\PYZhy{}}\PY{l+m+mi}{46}\PY{o}{/}\PY{l+m+mi}{17}\PY{p}{,} \PY{o}{\PYZhy{}}\PY{l+m+mi}{15}\PY{o}{/}\PY{l+m+mi}{11}\PY{p}{,}  \PY{l+m+mi}{23}\PY{o}{/}\PY{l+m+mi}{49}\PY{p}{]}\PY{p}{,}
\PY{p}{[}\PY{o}{\PYZhy{}}\PY{l+m+mi}{16}\PY{o}{/}\PY{l+m+mi}{27}\PY{p}{,}    \PY{l+m+mi}{1}\PY{o}{/}\PY{l+m+mi}{7}\PY{p}{,}  \PY{o}{\PYZhy{}}\PY{l+m+mi}{5}\PY{o}{/}\PY{l+m+mi}{33}\PY{p}{,}     \PY{l+m+mi}{50}\PY{p}{,}  \PY{l+m+mi}{75}\PY{o}{/}\PY{l+m+mi}{61}\PY{p}{,}  \PY{l+m+mi}{68}\PY{o}{/}\PY{l+m+mi}{55}\PY{p}{,}   \PY{l+m+mi}{9}\PY{o}{/}\PY{l+m+mi}{10}\PY{p}{,} \PY{o}{\PYZhy{}}\PY{l+m+mi}{88}\PY{o}{/}\PY{l+m+mi}{47}\PY{p}{,} \PY{o}{\PYZhy{}}\PY{l+m+mi}{82}\PY{o}{/}\PY{l+m+mi}{69}\PY{p}{,} \PY{o}{\PYZhy{}}\PY{l+m+mi}{15}\PY{o}{/}\PY{l+m+mi}{49}\PY{p}{]}\PY{p}{,}
\PY{p}{[} \PY{l+m+mi}{35}\PY{o}{/}\PY{l+m+mi}{27}\PY{p}{,}  \PY{o}{\PYZhy{}}\PY{l+m+mi}{7}\PY{o}{/}\PY{l+m+mi}{92}\PY{p}{,}   \PY{l+m+mi}{1}\PY{o}{/}\PY{l+m+mi}{46}\PY{p}{,}  \PY{o}{\PYZhy{}}\PY{l+m+mi}{13}\PY{o}{/}\PY{l+m+mi}{2}\PY{p}{,}  \PY{l+m+mi}{27}\PY{o}{/}\PY{l+m+mi}{34}\PY{p}{,} \PY{o}{\PYZhy{}}\PY{l+m+mi}{69}\PY{o}{/}\PY{l+m+mi}{65}\PY{p}{,}   \PY{l+m+mi}{49}\PY{o}{/}\PY{l+m+mi}{2}\PY{p}{,}   \PY{l+m+mi}{92}\PY{o}{/}\PY{l+m+mi}{3}\PY{p}{,}  \PY{l+m+mi}{60}\PY{o}{/}\PY{l+m+mi}{61}\PY{p}{,}  \PY{l+m+mi}{26}\PY{o}{/}\PY{l+m+mi}{15}\PY{p}{]}\PY{p}{,}
\PY{p}{[} \PY{l+m+mi}{13}\PY{o}{/}\PY{l+m+mi}{70}\PY{p}{,}      \PY{l+m+mi}{0}\PY{p}{,} \PY{o}{\PYZhy{}}\PY{l+m+mi}{52}\PY{o}{/}\PY{l+m+mi}{23}\PY{p}{,} \PY{o}{\PYZhy{}}\PY{l+m+mi}{92}\PY{o}{/}\PY{l+m+mi}{73}\PY{p}{,}   \PY{o}{\PYZhy{}}\PY{l+m+mi}{1}\PY{o}{/}\PY{l+m+mi}{2}\PY{p}{,}  \PY{l+m+mi}{41}\PY{o}{/}\PY{l+m+mi}{23}\PY{p}{,}  \PY{o}{\PYZhy{}}\PY{l+m+mi}{1}\PY{o}{/}\PY{l+m+mi}{83}\PY{p}{,} \PY{o}{\PYZhy{}}\PY{l+m+mi}{36}\PY{o}{/}\PY{l+m+mi}{11}\PY{p}{,}  \PY{l+m+mi}{11}\PY{o}{/}\PY{l+m+mi}{96}\PY{p}{,} \PY{o}{\PYZhy{}}\PY{l+m+mi}{29}\PY{o}{/}\PY{l+m+mi}{47}\PY{p}{]}\PY{p}{,}
\PY{p}{[}  \PY{l+m+mi}{10}\PY{o}{/}\PY{l+m+mi}{7}\PY{p}{,}  \PY{l+m+mi}{15}\PY{o}{/}\PY{l+m+mi}{41}\PY{p}{,} \PY{o}{\PYZhy{}}\PY{l+m+mi}{97}\PY{o}{/}\PY{l+m+mi}{99}\PY{p}{,}  \PY{l+m+mi}{69}\PY{o}{/}\PY{l+m+mi}{16}\PY{p}{,}   \PY{l+m+mi}{38}\PY{o}{/}\PY{l+m+mi}{5}\PY{p}{,} \PY{o}{\PYZhy{}}\PY{l+m+mi}{89}\PY{o}{/}\PY{l+m+mi}{67}\PY{p}{,} \PY{o}{\PYZhy{}}\PY{l+m+mi}{43}\PY{o}{/}\PY{l+m+mi}{19}\PY{p}{,} \PY{o}{\PYZhy{}}\PY{l+m+mi}{81}\PY{o}{/}\PY{l+m+mi}{94}\PY{p}{,}   \PY{l+m+mi}{4}\PY{o}{/}\PY{l+m+mi}{45}\PY{p}{,}  \PY{l+m+mi}{77}\PY{o}{/}\PY{l+m+mi}{79}\PY{p}{]}\PY{p}{,}
\PY{p}{[}\PY{o}{\PYZhy{}}\PY{l+m+mi}{45}\PY{o}{/}\PY{l+m+mi}{17}\PY{p}{,}  \PY{l+m+mi}{45}\PY{o}{/}\PY{l+m+mi}{97}\PY{p}{,}  \PY{l+m+mi}{59}\PY{o}{/}\PY{l+m+mi}{58}\PY{p}{,}   \PY{l+m+mi}{8}\PY{o}{/}\PY{l+m+mi}{67}\PY{p}{,} \PY{o}{\PYZhy{}}\PY{l+m+mi}{39}\PY{o}{/}\PY{l+m+mi}{71}\PY{p}{,} \PY{o}{\PYZhy{}}\PY{l+m+mi}{25}\PY{o}{/}\PY{l+m+mi}{44}\PY{p}{,}  \PY{l+m+mi}{97}\PY{o}{/}\PY{l+m+mi}{28}\PY{p}{,}   \PY{o}{\PYZhy{}}\PY{l+m+mi}{7}\PY{o}{/}\PY{l+m+mi}{2}\PY{p}{,}    \PY{l+m+mi}{7}\PY{o}{/}\PY{l+m+mi}{5}\PY{p}{,}  \PY{l+m+mi}{43}\PY{o}{/}\PY{l+m+mi}{17}\PY{p}{]}\PY{p}{,}
\PY{p}{[} \PY{l+m+mi}{81}\PY{o}{/}\PY{l+m+mi}{74}\PY{p}{,} \PY{o}{\PYZhy{}}\PY{l+m+mi}{97}\PY{o}{/}\PY{l+m+mi}{76}\PY{p}{,}  \PY{l+m+mi}{94}\PY{o}{/}\PY{l+m+mi}{15}\PY{p}{,}  \PY{l+m+mi}{29}\PY{o}{/}\PY{l+m+mi}{13}\PY{p}{,}   \PY{l+m+mi}{7}\PY{o}{/}\PY{l+m+mi}{34}\PY{p}{,} \PY{o}{\PYZhy{}}\PY{l+m+mi}{50}\PY{o}{/}\PY{l+m+mi}{93}\PY{p}{,}  \PY{l+m+mi}{53}\PY{o}{/}\PY{l+m+mi}{35}\PY{p}{,}  \PY{l+m+mi}{44}\PY{o}{/}\PY{l+m+mi}{17}\PY{p}{,}  \PY{o}{\PYZhy{}}\PY{l+m+mi}{6}\PY{o}{/}\PY{l+m+mi}{89}\PY{p}{,} \PY{o}{\PYZhy{}}\PY{l+m+mi}{78}\PY{o}{/}\PY{l+m+mi}{37}\PY{p}{]}\PY{p}{]}\PY{p}{)}
\end{Verbatim}
\end{tcolorbox}

    \begin{tcolorbox}[breakable, size=fbox, boxrule=1pt, pad at break*=1mm,colback=cellbackground, colframe=cellborder]
\prompt{In}{incolor}{2}{\boxspacing}
\begin{Verbatim}[commandchars=\\\{\}]
\PY{n}{c}
\end{Verbatim}
\end{tcolorbox}

            \begin{tcolorbox}[breakable, size=fbox, boxrule=.5pt, pad at break*=1mm, opacityfill=0]
\prompt{Out}{outcolor}{2}{\boxspacing}
\begin{Verbatim}[commandchars=\\\{\}]
(91/53, 1/100, 1/100, 1/100, 1/100, 1/100, 1/9, 3/44, 17/75, 1/100)
\end{Verbatim}
\end{tcolorbox}
        
    \begin{tcolorbox}[breakable, size=fbox, boxrule=1pt, pad at break*=1mm,colback=cellbackground, colframe=cellborder]
\prompt{In}{incolor}{3}{\boxspacing}
\begin{Verbatim}[commandchars=\\\{\}]
\PY{n}{Y}
\end{Verbatim}
\end{tcolorbox}

            \begin{tcolorbox}[breakable, size=fbox, boxrule=.5pt, pad at break*=1mm, opacityfill=0]
\prompt{Out}{outcolor}{3}{\boxspacing}
\begin{Verbatim}[commandchars=\\\{\}]
(5/68, 1/100, 29/36, 37/14, 67/18, 1/100, 11/21, 1/100, 44/83, 5/38)
\end{Verbatim}
\end{tcolorbox}
        
    \begin{tcolorbox}[breakable, size=fbox, boxrule=1pt, pad at break*=1mm,colback=cellbackground, colframe=cellborder]
\prompt{In}{incolor}{4}{\boxspacing}
\begin{Verbatim}[commandchars=\\\{\}]
\PY{c+c1}{\PYZsh{} here, A has one \PYZdq{}0\PYZdq{} entry}
\PY{n}{A}
\end{Verbatim}
\end{tcolorbox}

            \begin{tcolorbox}[breakable, size=fbox, boxrule=.5pt, pad at break*=1mm, opacityfill=0]
\prompt{Out}{outcolor}{4}{\boxspacing}
\begin{Verbatim}[commandchars=\\\{\}]
[ 13/17  84/95 -39/46  72/65 -45/16  -43/6  23/57  11/15  64/97 -49/26]
[  82/9 -73/42 -55/58   -8/5  57/23  53/51  77/19  65/28   32/3 -11/21]
[ 41/14  100/3 -86/75   -4/3  87/65 -97/82  34/81      6      1 -96/17]
[  97/2  -23/5 -26/45  13/28 -21/19 -32/13 -47/43 -46/17 -15/11  23/49]
[-16/27    1/7  -5/33     50  75/61  68/55   9/10 -88/47 -82/69 -15/49]
[ 35/27  -7/92   1/46  -13/2  27/34 -69/65   49/2   92/3  60/61  26/15]
[ 13/70      0 -52/23 -92/73   -1/2  41/23  -1/83 -36/11  11/96 -29/47]
[  10/7  15/41 -97/99  69/16   38/5 -89/67 -43/19 -81/94   4/45  77/79]
[-45/17  45/97  59/58   8/67 -39/71 -25/44  97/28   -7/2    7/5  43/17]
[ 81/74 -97/76  94/15  29/13   7/34 -50/93  53/35  44/17  -6/89 -78/37]
\end{Verbatim}
\end{tcolorbox}
        
    \begin{tcolorbox}[breakable, size=fbox, boxrule=1pt, pad at break*=1mm,colback=cellbackground, colframe=cellborder]
\prompt{In}{incolor}{5}{\boxspacing}
\begin{Verbatim}[commandchars=\\\{\}]
\PY{n}{LP} \PY{o}{=} \PY{n}{MixedIntegerLinearProgram}\PY{p}{(}\PY{n}{maximization}\PY{o}{=}\PY{k+kc}{True}\PY{p}{,}\PY{n}{solver}\PY{o}{=}\PY{l+s+s2}{\PYZdq{}}\PY{l+s+s2}{GLPK}\PY{l+s+s2}{\PYZdq{}}\PY{p}{)}
\PY{n}{x} \PY{o}{=} \PY{n}{LP}\PY{o}{.}\PY{n}{new\PYZus{}variable}\PY{p}{(}\PY{n}{nonnegative}\PY{o}{=}\PY{k+kc}{True}\PY{p}{)}
\PY{n}{LP}\PY{o}{.}\PY{n}{add\PYZus{}constraint}\PY{p}{(}\PY{n}{A}\PY{o}{*}\PY{n}{x} \PY{o}{\PYZlt{}}\PY{o}{=} \PY{n}{Y}\PY{p}{)}
\PY{n}{LP}\PY{o}{.}\PY{n}{set\PYZus{}objective}\PY{p}{(}\PY{n}{c}\PY{p}{)}
\end{Verbatim}
\end{tcolorbox}

    \begin{tcolorbox}[breakable, size=fbox, boxrule=1pt, pad at break*=1mm,colback=cellbackground, colframe=cellborder]
\prompt{In}{incolor}{6}{\boxspacing}
\begin{Verbatim}[commandchars=\\\{\}]
\PY{n}{LP}\PY{o}{.}\PY{n}{solver\PYZus{}parameter}\PY{p}{(}\PY{l+s+s2}{\PYZdq{}}\PY{l+s+s2}{simplex\PYZus{}or\PYZus{}intopt}\PY{l+s+s2}{\PYZdq{}}\PY{p}{,} \PY{l+s+s2}{\PYZdq{}}\PY{l+s+s2}{simplex\PYZus{}only}\PY{l+s+s2}{\PYZdq{}}\PY{p}{)}
\PY{n}{LP}\PY{o}{.}\PY{n}{solve}\PY{p}{(}\PY{p}{)}
\end{Verbatim}
\end{tcolorbox}

            \begin{tcolorbox}[breakable, size=fbox, boxrule=.5pt, pad at break*=1mm, opacityfill=0]
\prompt{Out}{outcolor}{6}{\boxspacing}
\begin{Verbatim}[commandchars=\\\{\}]
0.04203870150737101
\end{Verbatim}
\end{tcolorbox}
        
    \begin{tcolorbox}[breakable, size=fbox, boxrule=1pt, pad at break*=1mm,colback=cellbackground, colframe=cellborder]
\prompt{In}{incolor}{7}{\boxspacing}
\begin{Verbatim}[commandchars=\\\{\}]
\PY{n}{b} \PY{o}{=} \PY{n}{LP}\PY{o}{.}\PY{n}{get\PYZus{}backend}\PY{p}{(}\PY{p}{)}
\end{Verbatim}
\end{tcolorbox}

    \begin{tcolorbox}[breakable, size=fbox, boxrule=1pt, pad at break*=1mm,colback=cellbackground, colframe=cellborder]
\prompt{In}{incolor}{8}{\boxspacing}
\begin{Verbatim}[commandchars=\\\{\}]
\PY{n}{basic\PYZus{}vars} \PY{o}{=} \PY{p}{[}\PY{p}{(}\PY{n}{i}\PY{o}{+}\PY{l+m+mi}{1}\PY{p}{)} \PY{k}{for} \PY{n}{i} \PY{o+ow}{in} \PY{n+nb}{range}\PY{p}{(}\PY{n}{b}\PY{o}{.}\PY{n}{ncols}\PY{p}{(}\PY{p}{)}\PY{p}{)} \PY{k}{if} \PY{n}{b}\PY{o}{.}\PY{n}{is\PYZus{}variable\PYZus{}basic}\PY{p}{(}\PY{n}{i}\PY{p}{)}\PY{p}{]}\PY{o}{+}\PY{p}{[}\PY{p}{(}\PY{n}{b}\PY{o}{.}\PY{n}{nrows}\PY{p}{(}\PY{p}{)}\PY{o}{+}\PY{n}{j}\PY{o}{+}\PY{l+m+mi}{1}\PY{p}{)} \PY{k}{for} \PY{n}{j} \PY{o+ow}{in} \PY{n+nb}{range}\PY{p}{(}\PY{n}{b}\PY{o}{.}\PY{n}{nrows}\PY{p}{(}\PY{p}{)}\PY{p}{)} \PY{k}{if} \PY{n}{b}\PY{o}{.}\PY{n}{is\PYZus{}slack\PYZus{}variable\PYZus{}basic}\PY{p}{(}\PY{n}{j}\PY{p}{)}\PY{p}{]}
\end{Verbatim}
\end{tcolorbox}

    \begin{tcolorbox}[breakable, size=fbox, boxrule=1pt, pad at break*=1mm,colback=cellbackground, colframe=cellborder]
\prompt{In}{incolor}{9}{\boxspacing}
\begin{Verbatim}[commandchars=\\\{\}]
\PY{n}{LP\PYZus{}A} \PY{o}{=} \PY{p}{[}\PY{p}{]}
\PY{n}{LP\PYZus{}Y} \PY{o}{=} \PY{p}{[}\PY{p}{]}
\end{Verbatim}
\end{tcolorbox}

    \begin{tcolorbox}[breakable, size=fbox, boxrule=1pt, pad at break*=1mm,colback=cellbackground, colframe=cellborder]
\prompt{In}{incolor}{10}{\boxspacing}
\begin{Verbatim}[commandchars=\\\{\}]
    \PY{k}{for} \PY{n}{i} \PY{o+ow}{in} \PY{n+nb}{range}\PY{p}{(}\PY{n+nb}{len}\PY{p}{(}\PY{n}{LP}\PY{o}{.}\PY{n}{constraints}\PY{p}{(}\PY{p}{)}\PY{p}{)}\PY{p}{)}\PY{p}{:}
        
        \PY{n}{A\PYZus{}list} \PY{o}{=} \PY{p}{[}\PY{p}{]}

        \PY{k}{for} \PY{n}{j} \PY{o+ow}{in} \PY{n+nb}{range}\PY{p}{(}\PY{n+nb}{len}\PY{p}{(}\PY{n}{LP}\PY{o}{.}\PY{n}{constraints}\PY{p}{(}\PY{p}{)}\PY{p}{[}\PY{n}{i}\PY{p}{]}\PY{p}{[}\PY{l+m+mi}{1}\PY{p}{]}\PY{p}{[}\PY{l+m+mi}{1}\PY{p}{]}\PY{p}{)}\PY{p}{)}\PY{p}{:}
            
            \PY{n}{A\PYZus{}list}\PY{o}{.}\PY{n}{append}\PY{p}{(}\PY{n}{Rational}\PY{p}{(}\PY{n}{LP}\PY{o}{.}\PY{n}{constraints}\PY{p}{(}\PY{p}{)}\PY{p}{[}\PY{n}{i}\PY{p}{]}\PY{p}{[}\PY{l+m+mi}{1}\PY{p}{]}\PY{p}{[}\PY{l+m+mi}{1}\PY{p}{]}\PY{p}{[}\PY{o}{\PYZhy{}}\PY{p}{(}\PY{n}{j}\PY{o}{+}\PY{l+m+mi}{1}\PY{p}{)}\PY{p}{]}\PY{p}{)}\PY{p}{)}
            
            
        
        \PY{k}{if} \PY{n}{A\PYZus{}list} \PY{o}{!=} \PY{p}{[}\PY{p}{]}\PY{p}{:}
            \PY{n}{LP\PYZus{}A}\PY{o}{.}\PY{n}{append}\PY{p}{(}\PY{n}{A\PYZus{}list}\PY{p}{)}
            
        
        
        \PY{k}{if} \PY{n}{Rational}\PY{p}{(}\PY{n}{LP}\PY{o}{.}\PY{n}{constraints}\PY{p}{(}\PY{p}{)}\PY{p}{[}\PY{n}{i}\PY{p}{]}\PY{p}{[}\PY{l+m+mi}{2}\PY{p}{]}\PY{p}{)}\PY{o}{!=} \PY{l+m+mi}{0}\PY{p}{:}
            \PY{n}{LP\PYZus{}Y}\PY{o}{.}\PY{n}{append}\PY{p}{(}\PY{n}{Rational}\PY{p}{(}\PY{n}{LP}\PY{o}{.}\PY{n}{constraints}\PY{p}{(}\PY{p}{)}\PY{p}{[}\PY{n}{i}\PY{p}{]}\PY{p}{[}\PY{l+m+mi}{2}\PY{p}{]}\PY{p}{)}\PY{p}{)}
            
\end{Verbatim}
\end{tcolorbox}

    \begin{tcolorbox}[breakable, size=fbox, boxrule=1pt, pad at break*=1mm,colback=cellbackground, colframe=cellborder]
\prompt{In}{incolor}{11}{\boxspacing}
\begin{Verbatim}[commandchars=\\\{\}]
\PY{c+c1}{\PYZsh{} after construction, LP\PYZus{}A omits the \PYZdq{}0\PYZdq{} entry}
\PY{n}{LP\PYZus{}A}
\end{Verbatim}
\end{tcolorbox}

            \begin{tcolorbox}[breakable, size=fbox, boxrule=.5pt, pad at break*=1mm, opacityfill=0]
\prompt{Out}{outcolor}{11}{\boxspacing}
\begin{Verbatim}[commandchars=\\\{\}]
[[13/17, 84/95, -39/46, 72/65, -45/16, -43/6, 23/57, 11/15, 64/97, -49/26],
 [82/9, -73/42, -55/58, -8/5, 57/23, 53/51, 77/19, 65/28, 32/3, -11/21],
 [41/14, 100/3, -86/75, -4/3, 87/65, -97/82, 34/81, 6, 1, -96/17],
 [97/2, -23/5, -26/45, 13/28, -21/19, -32/13, -47/43, -46/17, -15/11, 23/49],
 [-16/27, 1/7, -5/33, 50, 75/61, 68/55, 9/10, -88/47, -82/69, -15/49],
 [35/27, -7/92, 1/46, -13/2, 27/34, -69/65, 49/2, 92/3, 60/61, 26/15],
 [13/70, -52/23, -92/73, -1/2, 41/23, -1/83, -36/11, 11/96, -29/47],
 [10/7, 15/41, -97/99, 69/16, 38/5, -89/67, -43/19, -81/94, 4/45, 77/79],
 [-45/17, 45/97, 59/58, 8/67, -39/71, -25/44, 97/28, -7/2, 7/5, 43/17],
 [81/74, -97/76, 94/15, 29/13, 7/34, -50/93, 53/35, 44/17, -6/89, -78/37]]
\end{Verbatim}
\end{tcolorbox}
        
    \begin{tcolorbox}[breakable, size=fbox, boxrule=1pt, pad at break*=1mm,colback=cellbackground, colframe=cellborder]
\prompt{In}{incolor}{12}{\boxspacing}
\begin{Verbatim}[commandchars=\\\{\}]
\PY{n}{A}
\end{Verbatim}
\end{tcolorbox}

            \begin{tcolorbox}[breakable, size=fbox, boxrule=.5pt, pad at break*=1mm, opacityfill=0]
\prompt{Out}{outcolor}{12}{\boxspacing}
\begin{Verbatim}[commandchars=\\\{\}]
[ 13/17  84/95 -39/46  72/65 -45/16  -43/6  23/57  11/15  64/97 -49/26]
[  82/9 -73/42 -55/58   -8/5  57/23  53/51  77/19  65/28   32/3 -11/21]
[ 41/14  100/3 -86/75   -4/3  87/65 -97/82  34/81      6      1 -96/17]
[  97/2  -23/5 -26/45  13/28 -21/19 -32/13 -47/43 -46/17 -15/11  23/49]
[-16/27    1/7  -5/33     50  75/61  68/55   9/10 -88/47 -82/69 -15/49]
[ 35/27  -7/92   1/46  -13/2  27/34 -69/65   49/2   92/3  60/61  26/15]
[ 13/70      0 -52/23 -92/73   -1/2  41/23  -1/83 -36/11  11/96 -29/47]
[  10/7  15/41 -97/99  69/16   38/5 -89/67 -43/19 -81/94   4/45  77/79]
[-45/17  45/97  59/58   8/67 -39/71 -25/44  97/28   -7/2    7/5  43/17]
[ 81/74 -97/76  94/15  29/13   7/34 -50/93  53/35  44/17  -6/89 -78/37]
\end{Verbatim}
\end{tcolorbox}
        
    \begin{tcolorbox}[breakable, size=fbox, boxrule=1pt, pad at break*=1mm,colback=cellbackground, colframe=cellborder]
\prompt{In}{incolor}{13}{\boxspacing}
\begin{Verbatim}[commandchars=\\\{\}]
    \PY{n}{LP\PYZus{}c} \PY{o}{=} \PY{p}{[}\PY{p}{]}

    \PY{k}{for} \PY{n}{j} \PY{o+ow}{in} \PY{n+nb}{range}\PY{p}{(}\PY{n}{LP}\PY{o}{.}\PY{n}{number\PYZus{}of\PYZus{}variables}\PY{p}{(}\PY{p}{)}\PY{p}{)}\PY{p}{:}
        \PY{k}{if} \PY{n}{b}\PY{o}{.}\PY{n}{objective\PYZus{}coefficient}\PY{p}{(}\PY{n}{j}\PY{p}{)} \PY{o}{!=} \PY{p}{[}\PY{p}{]}\PY{p}{:}
            \PY{n}{LP\PYZus{}c}\PY{o}{.}\PY{n}{append}\PY{p}{(}\PY{n}{Rational}\PY{p}{(}\PY{n}{b}\PY{o}{.}\PY{n}{objective\PYZus{}coefficient}\PY{p}{(}\PY{n}{j}\PY{p}{)}\PY{p}{)}\PY{p}{)}
            \PY{c+c1}{\PYZsh{}print(c)}
\end{Verbatim}
\end{tcolorbox}

    \begin{tcolorbox}[breakable, size=fbox, boxrule=1pt, pad at break*=1mm,colback=cellbackground, colframe=cellborder]
\prompt{In}{incolor}{14}{\boxspacing}
\begin{Verbatim}[commandchars=\\\{\}]
\PY{n}{P} \PY{o}{=} \PY{n}{InteractiveLPProblemStandardForm}\PY{p}{(}\PY{n}{LP\PYZus{}A}\PY{p}{,} \PY{n}{LP\PYZus{}Y}\PY{p}{,} \PY{n}{LP\PYZus{}c}\PY{p}{)}
\end{Verbatim}
\end{tcolorbox}

    \begin{Verbatim}[commandchars=\\\{\}]

        ---------------------------------------------------------------------------

        ValueError                                Traceback (most recent call last)

        <ipython-input-14-46bdd7721161> in <module>
    ----> 1 P = InteractiveLPProblemStandardForm(LP\_A, LP\_Y, LP\_c)
    

        /mnt/c/users/phili/SAGEMATH/sage-9.2/local/lib/python3.8/site-packages/sage/numerical/interactive\_simplex\_method.py in \_\_init\_\_(self, A, b, c, x, problem\_type, slack\_variables, auxiliary\_variable, base\_ring, is\_primal, objective\_name, objective\_constant\_term)
       1977             raise ValueError("problems in standard form must be of (negative) "
       1978                              "maximization type")
    -> 1979         super(InteractiveLPProblemStandardForm, self).\_\_init\_\_(
       1980             A, b, c, x,
       1981             problem\_type=problem\_type,


        /mnt/c/users/phili/SAGEMATH/sage-9.2/local/lib/python3.8/site-packages/sage/numerical/interactive\_simplex\_method.py in \_\_init\_\_(self, A, b, c, x, constraint\_type, variable\_type, problem\_type, base\_ring, is\_primal, objective\_constant\_term)
        646         """
        647         super(InteractiveLPProblem, self).\_\_init\_\_()
    --> 648         A = matrix(A)
        649         b = vector(b)
        650         c = vector(c)


        /mnt/c/users/phili/SAGEMATH/sage-9.2/local/lib/python3.8/site-packages/sage/matrix/constructor.pyx in sage.matrix.constructor.matrix (build/cythonized/sage/matrix/constructor.c:2529)()
        633     """
        634     immutable = kwds.pop('immutable', False)
    --> 635     M = MatrixArgs(*args, **kwds).matrix()
        636     if immutable:
        637         M.set\_immutable()


        /mnt/c/users/phili/SAGEMATH/sage-9.2/local/lib/python3.8/site-packages/sage/matrix/args.pyx in sage.matrix.args.MatrixArgs.matrix (build/cythonized/sage/matrix/args.c:7770)()
        650             True
        651         """
    --> 652         self.finalize()
        653 
        654         cdef Matrix M


        /mnt/c/users/phili/SAGEMATH/sage-9.2/local/lib/python3.8/site-packages/sage/matrix/args.pyx in sage.matrix.args.MatrixArgs.finalize (build/cythonized/sage/matrix/args.c:10107)()
        913             \# really need to look at the entries.
        914             if self.typ == MA\_ENTRIES\_SEQ\_SEQ:
    --> 915                 self.finalize\_seq\_seq()
        916             elif self.typ == MA\_ENTRIES\_SEQ\_FLAT:
        917                 self.finalize\_seq\_scalar()


        /mnt/c/users/phili/SAGEMATH/sage-9.2/local/lib/python3.8/site-packages/sage/matrix/args.pyx in sage.matrix.args.MatrixArgs.finalize\_seq\_seq (build/cythonized/sage/matrix/args.c:13547)()
       1129                 c += 1
       1130                 entries.append(entry)
    -> 1131             self.set\_ncols(c)
       1132 
       1133         self.set\_seq\_flat(entries)


        /mnt/c/users/phili/SAGEMATH/sage-9.2/local/lib/python3.8/site-packages/sage/matrix/args.pxd in sage.matrix.args.MatrixArgs.set\_ncols (build/cythonized/sage/matrix/args.c:17815)()
         90         cdef long p = self.ncols
         91         if p != -1 and p != n:
    ---> 92             raise ValueError(f"inconsistent number of columns: should be \{p\} but got \{n\}")
         93         self.ncols = n
         94 


        ValueError: inconsistent number of columns: should be 10 but got 9

    \end{Verbatim}

    \begin{tcolorbox}[breakable, size=fbox, boxrule=1pt, pad at break*=1mm,colback=cellbackground, colframe=cellborder]
\prompt{In}{incolor}{ }{\boxspacing}
\begin{Verbatim}[commandchars=\\\{\}]
\PY{n}{D} \PY{o}{=} \PY{n}{P}\PY{o}{.}\PY{n}{dictionary}\PY{p}{(}\PY{o}{*}\PY{n}{basic\PYZus{}vars}\PY{p}{)}
\end{Verbatim}
\end{tcolorbox}

    \begin{tcolorbox}[breakable, size=fbox, boxrule=1pt, pad at break*=1mm,colback=cellbackground, colframe=cellborder]
\prompt{In}{incolor}{ }{\boxspacing}
\begin{Verbatim}[commandchars=\\\{\}]
\PY{n}{D}\PY{o}{.}\PY{n}{is\PYZus{}optimal}\PY{p}{(}\PY{p}{)}
\end{Verbatim}
\end{tcolorbox}

    \begin{tcolorbox}[breakable, size=fbox, boxrule=1pt, pad at break*=1mm,colback=cellbackground, colframe=cellborder]
\prompt{In}{incolor}{ }{\boxspacing}
\begin{Verbatim}[commandchars=\\\{\}]
\PY{n}{D}\PY{o}{.}\PY{n}{basic\PYZus{}solution}\PY{p}{(}\PY{p}{)}
\end{Verbatim}
\end{tcolorbox}


    % Add a bibliography block to the postdoc
    
    
    
\end{document}
